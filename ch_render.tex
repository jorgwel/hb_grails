% \chapter{Arquitectura y comandos b\'asicos}
\chapter{M\'etodos de renderizado}

% Define block styles
\tikzstyle{decision} = [diamond, draw, fill=blue!20, 
    text width=4.5em, text badly centered, node distance=3cm, inner sep=0pt]
\tikzstyle{block} = [rectangle, draw, fill=blue!20, 
    text width=5em, text centered, rounded corners, minimum height=4em]
\tikzstyle{line} = [draw, -latex']
\tikzstyle{cloud} = [draw, ellipse,fill=white, node distance=3cm,
    minimum height=2em]

    
\begin{figure}[ht!]
    \begin{tikzpicture}[node distance = 2cm, auto]

        % Place nodes
        \node [block] (init) {controllers};
        \node [cloud, left of=init, node distance=5.1cm] (views) {views};
        \node [cloud, right of=init, node distance=3.4cm] (services) {services};    
        \node [cloud, below of=views] (views2) {views};
    %     \node [block, below of=init] (identify) {identify candidate models};
    %     \node [block, below of=identify] (evaluate) {evaluate candidate models};
    %     \node [block, left of=evaluate, node distance=3cm] (update) {update model};
    %     \node [decision, below of=evaluate] (decide) {is best candidate better?};
    %     \node [block, below of=decide, node distance=3cm] (stop) {stop};

        % Draw edges
        \path [line] (views) -- node [auto] {1.-jquery, canjs, form} (init);
    %     \path [line] (views) -- (views);
        \path [line] (init) -- node [auto] {2.-invoke} (services);
        \path [line] (init) -- node [auto] {3.-render response: views, json, txt} (views2);
        
    %     \path [line] (init) -- (identify);
    %     \path [line] (identify) -- (evaluate);
    %     \path [line] (evaluate) -- (decide);
    %     \path [line] (decide) -| node [near start] {yes} (update);
    %     \path [line] (update) |- (identify);
    %     \path [line] (decide) -- node {no}(stop);
    %     \path [line,dashed] (expert) -- (init);
    %     \path [line,dashed] (system) -- (init);
    %     \path [line,dashed] (system) |- (evaluate);
    \end{tikzpicture}
    \caption{Flujo de datos b\'asico de Grails}
\end{figure}


% \begin{tikzpicture}[node distance = 2cm, auto]
%     
%     \node [block] (init) {initialize model};
%     \node [cloud, left of=init] (expert) {expert};
%     \node [cloud, right of=init] (system) {system};
%     \node [block, below of=init] (identify) {identify candidate models};
%     \node [block, below of=identify] (evaluate) {evaluate candidate models};
%     \node [block, left of=evaluate, node distance=3cm] (update) {update model};
%     \node [decision, below of=evaluate] (decide) {is best candidate better?};
%     \node [block, below of=decide, node distance=3cm] (stop) {stop};
%     
%     \path [line] (init) -- (identify);
%     \path [line] (identify) -- (evaluate);
%     \path [line] (evaluate) -- (decide);
%     \path [line] (decide) -| node [near start] {yes} (update);
%     \path [line] (update) |- (identify);
%     \path [line] (decide) -- node {no}(stop);
%     \path [line,dashed] (expert) -- (init);
%     \path [line,dashed] (system) -- (init);
%     \path [line,dashed] (system) |- (evaluate);
% \end{tikzpicture}
El resultado enviado de los controllers a las vistas puede ir en varios formatos. Siempre se hace a trav\'es de la instrucci\'ion \textit{render}:
\begin{itemize}

    \item \textbf{Renderizar texto}    
        \begin{lstlisting}[gobble=11] 
            render 'Welcome to the gTunes store!'
        \end{lstlisting}

    \item \textbf{Solicitar ayuda de un comando}
        \begin{lstlisting}[gobble=11]
            grails help NOMBRE_DE_COMANDO
        \end{lstlisting}

    \item \textbf{Crear aplicaci\'on de Grails}
        \begin{lstlisting}[gobble=11]
            grails create-app NOMBRE_DE_APP
        \end{lstlisting}


    \item \textbf{Crear un controller + sus pruebas + sus vistas}
        \begin{lstlisting}[gobble=11]
            grails create-controller [PAQUETE]NOMBRE
        \end{lstlisting}

    \item \textbf{Ejecutar pruebas de la aplicaci\'on}
        \begin{lstlisting}[gobble=11]
            grails test-app
        \end{lstlisting}

    \item \textbf{Ejecutar la aplicaci\'on}
        \begin{lstlisting}[gobble=11]
            grails run-app
        \end{lstlisting}
        
\end{itemize}
