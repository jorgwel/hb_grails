% \chapter{Arquitectura y comandos b\'asicos}
\chapter{Comandos m\'as comunes}

\begin{itemize}

    \item \textbf{Consultar versi\'on de Grails}    
        \\Devolver\'a una l\'inea de texto describiendo la versi\'on de grails.
        \begin{lstlisting}[gobble=11] 
            grails -version
        \end{lstlisting}
        

    \item \textbf{Solicitar ayuda de un comando}
        \\Devolver\'a informaci\'on (no tan detallada) sobre el comando y sus argumentos.
        \begin{lstlisting}[gobble=11]
            grails help NOMBRE_DE_COMANDO
        \end{lstlisting}
        

    \item \textbf{Crear aplicaci\'on de Grails}
        \\Genera toda la estructura de archivos de una aplicaci\'on de Grails.
        \begin{lstlisting}[gobble=11]
            grails create-app NOMBRE_DE_APP
        \end{lstlisting}
        

    \item \textbf{Crear un controller + sus pruebas + sus vistas}
        \\Genera un controlador de Grails as\'i como archivos de prueba y \textit{gsp's}. Tanto los archivos de prueba como los \textit{gsp's} se generan vac\'ios.
        \begin{lstlisting}[gobble=11]
            grails create-controller [PAQUETE]NOMBRE
        \end{lstlisting}
        

    \item \textbf{Ejecutar pruebas de la aplicaci\'on}
        \\Ejecuta todas las pruebas de la aplicaci\'on, poniendo los resultados en \textit{test/reports}
        \begin{lstlisting}[gobble=11]
            grails test-app
        \end{lstlisting}
        

    \item \textbf{Ejecutar la aplicaci\'on}
        \\Ejecuta la aplicaci\'on. Si ENVIRONMENT no se especifica, el default a usar ser\'a el de desarrollo (development).
        \begin{lstlisting}[gobble=11]
            grails [ENVIRONMENT] run-app
        \end{lstlisting}
        
        
    \item \textbf{Crear clase de dominio}
        \\Se encarga de crear una nueva clase de dominio y adem\'as genera un archivo donde pueden implantarse sus pruebas.
        \begin{lstlisting}[gobble=11]
            grails create-domain-class [PAQUETE]NOMBRE
        \end{lstlisting}        
        
    \item \textbf{Abrir reporte de pruebas}
        \\Se encarga de abrir el reporte con los resultados de las pruebas con formato HTML en el navegador predeterminado.
        \begin{lstlisting}[gobble=11]
            grails open test-report
        \end{lstlisting}            
\end{itemize}
