\documentclass[10pt,oneside]{book} %oneside es para eliminar las hojas en blanco entre cada capítulo
\usepackage[pdftex]{graphicx}
% letterpaper
\usepackage[utf8]{inputenc}
\usepackage[spanish]{babel} % varias definiciones para el español (por ejemplo usa ''Índice'' en lugar de ''Contents'')
\usepackage{hyperref}
\usepackage{geometry}
\usepackage{multicol}
%Para mostrar c\'odigo fuente
\usepackage{listings}
% Libraries for diagrams
\usepackage{tikz}
\usetikzlibrary{shapes,arrows}
\usepackage{caption}
%%%%%%%%%%%%%%%%%%%%%%%%%%%%%

\lstset
{%
        %linewidth=\linewidth,  
        breaklines=true,
        tabsize=1, 
        showstringspaces=false%      
}
%\usepackage{lstautogobble}
%
\usepackage{mdwlist}
%\usepackage{enumitem}
%\usepackage{ragged2e}
%\addcontentsline{toc}{itemize}{Lista de informaci\'on}
\geometry{
    a5paper,
%     a6paper,
%     total={210mm,297mm},
    left=20mm,
    right=20mm,
    top=20mm,
    bottom=20mm,
    bindingoffset=0mm
}
\title{T\'itulo del libro}
\author{Jorge Bautista}


\begin{document}
    \frontmatter
    \maketitle
    %Prólogo del libro
    
\chapter{Pr\'ologo}
Este es un libro que menciona de manera breve todas las caracter\'isticas del framework \href{``http://www.grails.org/''}{'Grails'} descritas en el libro \textit{``The Definitive Guide To Grails''}. No se detiene a explicar detalles. Simplemente permite que el lector se asome a la tecnolog\'ia para comenzar a conocerla o para buscar posibles soluciones a problemas encontrados durante el desarrollo.


    
    \mainmatter
    %Cap\'itulo 1: 'B\'asicos'
    \chapter{Arquitectura y comandos b\'asicos}
Este es el cap\'itulo 1


    \include{ch2}
%     \backmatter
    \appendix    
    % \chapter{Arquitectura y comandos b\'asicos}
\chapter{Comandos m\'as comunes}

\begin{itemize}

    \item \textbf{Consultar versi\'on de Grails}    
        \begin{lstlisting}[gobble=11] 
            grails -version
        \end{lstlisting}

    \item \textbf{Solicitar ayuda de un comando}
        \begin{lstlisting}[gobble=11]
            grails help NOMBRE_DE_COMANDO
        \end{lstlisting}

    \item \textbf{Crear aplicaci\'on de Grails}
        \begin{lstlisting}[gobble=11]
            grails create-app NOMBRE_DE_APP
        \end{lstlisting}


    \item \textbf{Crear un controller + sus pruebas + sus vistas}
        \begin{lstlisting}[gobble=11]
            grails create-controller [PAQUETE]NOMBRE
        \end{lstlisting}

    \item \textbf{Ejecutar pruebas de la aplicaci\'on}
        \begin{lstlisting}[gobble=11]
            grails test-app
        \end{lstlisting}

    \item \textbf{Ejecutar la aplicaci\'on}
        \begin{lstlisting}[gobble=11]
            grails run-app
        \end{lstlisting}
        
\end{itemize}

    
    \listoffigures
    \listoftables
    %\nocite{*}
    \begin{thebibliography}{99}    
        \bibitem{bib-name1} article-description1
        \bibitem{bib-name2} article-description2
    \end{thebibliography}
\end{document}

%\documentclass[a4paper,10pt]{book}
%\usepackage[utf8]{inputenc}

%\begin{document}

%\end{document}
