\documentclass[10pt,oneside]{book} %oneside es para eliminar las hojas en blanco entre cada capítulo
\usepackage[pdftex]{graphicx}
% letterpaper
\usepackage[utf8]{inputenc}
%Para instalar babel en español, Fedora: sudo yum install texlive-collection-langspanish.noarch, Debian: sudo apt-get install texlive-lang-all
\usepackage[spanish]{babel} % varias definiciones para el español (por ejemplo usa ''Índice'' en lugar de ''Contents'')
\usepackage{hyperref}
\usepackage{geometry}
\usepackage{multicol}
%Para mostrar c\'odigo fuente
\usepackage{listings}

% Libraries for diagrams
\usepackage{tikz}
\usetikzlibrary{shapes,arrows}
\usepackage{caption}
%%%%%%%%%%%%%%%%%%%%%%%%%%%%%

\lstset
{%
        %linewidth=\linewidth,  
        breaklines=true,
        tabsize=1, 
        showstringspaces=false%      
}

%%%%%%%%%%DEFINICIONES DE NODOS%%%%%%%%%%
% Define block styles
\tikzstyle{decision} = [diamond, draw, fill=blue!20, 
    text width=4.5em, text badly centered, node distance=3cm, inner sep=0pt]
\tikzstyle{block} = [rectangle, draw, fill=blue!20, 
    text width=5em, text centered, rounded corners, minimum height=4em]
\tikzstyle{line} = [draw, -latex']
\tikzstyle{cloud} = [draw, ellipse,fill=white, node distance=3cm,
    minimum height=2em]
%%%%%%%%%%%%%%%%%%%%%%%%%%%%%%%%%%%%%%%%

%\usepackage{lstautogobble}
%
\usepackage{mdwlist}
%\usepackage{enumitem}
%\usepackage{ragged2e}
%\addcontentsline{toc}{itemize}{Lista de informaci\'on}
\geometry{
    a5paper,
%     a6paper,
%     total={210mm,297mm},
    left=20mm,
    right=20mm,
    top=20mm,
    bottom=20mm,
    bindingoffset=0mm
}
\title{T\'itulo del libro}
\author{Jorge Bautista}


\begin{document}
    \frontmatter
    \maketitle
    %Prólogo del libro
    
\chapter{Pr\'ologo}
Este es un libro que menciona de manera breve todas las caracter\'isticas del framework \href{``http://www.grails.org/''}{'Grails'} descritas en el libro \textit{``The Definitive Guide To Grails''}. No se detiene a explicar detalles. Simplemente permite que el lector se asome a la tecnolog\'ia para comenzar a conocerla o para buscar posibles soluciones a problemas encontrados durante el desarrollo.


      
    \mainmatter
    %Cap\'itulo 1: 'B\'asicos'
    % \chapter{Arquitectura y comandos b\'asicos}
\chapter{Introducci\'on a Grails}
\section{Arquitectura}
Grails envuelve a muchas tecnolog\'ias relacionados con \textit{java}, abstrayendo su complejidad al proporcionarnos configuraciones predeterminadas que cubren los casos de uso m\'as comunes. Estas configuraciones pueden ahorrarnos d\'ias o semanas de trabajo gracias a la integraci\'on con otras tecnolog\'ias.

Aqu\'i se presentan las tecnolog\'ias principales sobre las que se Grails est\'a montado:
%En este cap\'itulo se listan las tecnolog\'ias que usa grails y comandos b\'asicos para comenzar a usarlo. 

\begin{multicols}{2}
\begin{itemize}
  \item Hibernate
  \item Spring
  \item Sitemesh
  \item Tomcat
  \item H2
  \item Groovy
  \item Gant
  \item JEE
\end{itemize}
\end{multicols}

\begin{figure}[ht!]

    \includegraphics[width=100mm]{img/arch}
    \caption{Tecnolog\'ias que confirman la arquitectura de Grails}
    \label{arquitectura}

\end{figure}


\newpage
\section{Artefactos de Grails}
Grails consta de cuatro tipos de artefactos principales:
\begin{enumerate}
 \item Views
 \item Controllers
 \item Domains
 \item Services
\end{enumerate}

\subsection{Views}
Muestran informaci\'on al usuario de la aplicaci\'on. En Grails, las vistas existen en forma de archivos con extensi\'on \textit{gsp}. La informaci\'on es puesta en los \textit{gsp} a trav\'es de plantillas\footnote{Para los m\'as experimentados, llenar estas plantillas usan una sintaxis intermedia entre tags \textit{jstl} y tags de \textit{jsf}} y estas se llenan a trav\'es de controllers. 

\subsection{Controllers}
Se encargan de recibir peticiones desde las vistas, respondiendo con alguna de las siguientes formas:
\begin{enumerate}
 \item Renderizar una vista \textit{gsp},
 \item renderizar \textit{JSON},
 \item renderizar fragmentos \textit{HTML}
 \item etc.
\end{enumerate}

Asimismo los controllers deben lidiar con la l\'ogica de negocio de la aplicaci\'on, la cual puede colocarse dentro del mismo controller o en servicios inyectados\footnote{Checar Ap\'endice \ref{_depinject}}.

\subsection{Dominios}
Se encargan de describir el esquema de la base de datos a trav\'es de clases groovy. Grails se refiere al conjunto de estas clases y las funcionalidades agregadas por el framework\footnote{Cualquier funcionalidad relacionada con base de datos como CREATE, READ, UPDATE, DELETE y todas las que extienden de ellas} como \textit{GORM}.

\subsection{Servicios}
Se encargan de contener la l\'ogica de la aplicaci\'on y se conectan directamente con las clases de dominio.

Debido a la gran simplicidad de Grails, es com\'un que los desarrolladores escriban c\'odigo de acceso a base de datos y/o reglas de negocio dentro de los controladores. Y est\'a bien si as\'i les funciona, pero pierden las ventajas de algunas propiedades que los servicios tienen, por ejemplo:

\begin{itemize}
 \item Modularizaci\'on de c\'odigo
 \item Son transaccionales de forma predeterminada, es decir, si hay una operaci\'on a base de datos que necesite de varios pasos para completarse, y alguno de estos pasos falla, Grails ejecuta un rollback a todos los cambios.
\end{itemize}



\subsection{UrlMappings}
Es un archivo donde pueden describirse las \textit{URL's} que el sistema ofrece para ser accesibles al cliente. El \textit{UrlMappings} puede o no usarse, ya que las convenciones de Grails permiten que existan \textit{URL's} implicitas que podemos accesar aún si no han sido agregadas a este archivo.

\begin{figure}[ht!]
    \begin{tikzpicture}[node distance = 2cm, auto]

        % Place nodes
        \node [block] (init) {controllers};
        \node [cloud, above of=init, node distance=2.5cm] (urls) {UrlMappings \newline(optional)};
        \node [cloud, left of=init, node distance=4.3cm] (views) {views};        
        \node [cloud, right of=init, node distance=4.1cm] (services) {services};    
        %\node [cloud, below of=views] (views2) {views};
        \node [cloud, below of=services] (domains) {domains};

        % Draw edges
        \path [line] (views) -- node [auto] {1.1 jquery, form} (urls);
        \path [line] (urls) -- node [auto] {1.2 (controlador y acci\'on decodificados)} (init);
    %     \path [line] (views) -- (views);
        \path [line] (init) -- node [auto] {2.1 invoke} (services);
        \path [line] (init) -- node [auto] {4 html,json,txt} (views);
        \path [line] (init) -- node [auto] {2.2 invoke} (domains);
        \path [line] (services) -- node [auto] {3 invoke} (domains);
        
    \end{tikzpicture}
    \caption{Artefactos de Grails funcionando}
\end{figure}
    \chapter{Inicio r\'apido}

\begin{enumerate}
 \item Instala Grails y ponlo en el \textit{Path} de tu sistema operativo
 \item Verifica que Grails est\'e en el \textit{Path} ejecutando el siguiente comando desde una l\'inea de comandos y asegur\'andote de que devuelva la versi\'on instalada de Grails:
        \begin{lstlisting}[gobble=11]
            grails -version
        \end{lstlisting}
 \item Navega desde una l\'inea de comandos hacia el folder d\'onde generar\'as tu aplicaci\'on
 
 \item Para autogenerar una aplicaci\'on, ejecuta desde una l\'inea de comandos:
        \begin{lstlisting}[gobble=11]
            grails create-app NOMBRE_DE_APP
        \end{lstlisting}
        
 \item Para correr tu aplicaci\'on en el servidor web integrado, ejecuta desde una l\'inea de comandos y dentro del folder autogenerado por Grails:
        \begin{lstlisting}[gobble=11]
            grails run-app
        \end{lstlisting}
        
\item Accede a la direcci\'on:
        \begin{lstlisting}[gobble=11]
            http://localhost:8080/NOMBRE_DE_APP
        \end{lstlisting}       

\end{enumerate}



    % \chapter{Arquitectura y comandos b\'asicos}
\chapter{M\'etodos de renderizado}

% Define block styles
\tikzstyle{decision} = [diamond, draw, fill=blue!20, 
    text width=4.5em, text badly centered, node distance=3cm, inner sep=0pt]
\tikzstyle{block} = [rectangle, draw, fill=blue!20, 
    text width=5em, text centered, rounded corners, minimum height=4em]
\tikzstyle{line} = [draw, -latex']
\tikzstyle{cloud} = [draw, ellipse,fill=white, node distance=3cm,
    minimum height=2em]

    
\begin{figure}[ht!]
    \begin{tikzpicture}[node distance = 2cm, auto]

        % Place nodes
        \node [block] (init) {controllers};
        \node [cloud, left of=init, node distance=5.1cm] (views) {views};
        \node [cloud, right of=init, node distance=3.4cm] (services) {services};    
        \node [cloud, below of=views] (views2) {views};
    %     \node [block, below of=init] (identify) {identify candidate models};
    %     \node [block, below of=identify] (evaluate) {evaluate candidate models};
    %     \node [block, left of=evaluate, node distance=3cm] (update) {update model};
    %     \node [decision, below of=evaluate] (decide) {is best candidate better?};
    %     \node [block, below of=decide, node distance=3cm] (stop) {stop};

        % Draw edges
        \path [line] (views) -- node [auto] {1.-jquery, canjs, form} (init);
    %     \path [line] (views) -- (views);
        \path [line] (init) -- node [auto] {2.-invoke} (services);
        \path [line] (init) -- node [auto] {3.-render response: views, json, txt} (views2);
        
    %     \path [line] (init) -- (identify);
    %     \path [line] (identify) -- (evaluate);
    %     \path [line] (evaluate) -- (decide);
    %     \path [line] (decide) -| node [near start] {yes} (update);
    %     \path [line] (update) |- (identify);
    %     \path [line] (decide) -- node {no}(stop);
    %     \path [line,dashed] (expert) -- (init);
    %     \path [line,dashed] (system) -- (init);
    %     \path [line,dashed] (system) |- (evaluate);
    \end{tikzpicture}
    \caption{Flujo de datos b\'asico de Grails}
\end{figure}


% \begin{tikzpicture}[node distance = 2cm, auto]
%     
%     \node [block] (init) {initialize model};
%     \node [cloud, left of=init] (expert) {expert};
%     \node [cloud, right of=init] (system) {system};
%     \node [block, below of=init] (identify) {identify candidate models};
%     \node [block, below of=identify] (evaluate) {evaluate candidate models};
%     \node [block, left of=evaluate, node distance=3cm] (update) {update model};
%     \node [decision, below of=evaluate] (decide) {is best candidate better?};
%     \node [block, below of=decide, node distance=3cm] (stop) {stop};
%     
%     \path [line] (init) -- (identify);
%     \path [line] (identify) -- (evaluate);
%     \path [line] (evaluate) -- (decide);
%     \path [line] (decide) -| node [near start] {yes} (update);
%     \path [line] (update) |- (identify);
%     \path [line] (decide) -- node {no}(stop);
%     \path [line,dashed] (expert) -- (init);
%     \path [line,dashed] (system) -- (init);
%     \path [line,dashed] (system) |- (evaluate);
% \end{tikzpicture}
El resultado enviado de los controllers a las vistas puede ir en varios formatos. Siempre se hace a trav\'es de la instrucci\'ion \textit{render}:
\begin{itemize}

    \item \textbf{Renderizar texto}    
        \begin{lstlisting}[gobble=11] 
            render 'Welcome to the gTunes store!'
        \end{lstlisting}

    \item \textbf{Solicitar ayuda de un comando}
        \begin{lstlisting}[gobble=11]
            grails help NOMBRE_DE_COMANDO
        \end{lstlisting}

    \item \textbf{Crear aplicaci\'on de Grails}
        \begin{lstlisting}[gobble=11]
            grails create-app NOMBRE_DE_APP
        \end{lstlisting}


    \item \textbf{Crear un controller + sus pruebas + sus vistas}
        \begin{lstlisting}[gobble=11]
            grails create-controller [PAQUETE]NOMBRE
        \end{lstlisting}

    \item \textbf{Ejecutar pruebas de la aplicaci\'on}
        \begin{lstlisting}[gobble=11]
            grails test-app
        \end{lstlisting}

    \item \textbf{Ejecutar la aplicaci\'on}
        \begin{lstlisting}[gobble=11]
            grails run-app
        \end{lstlisting}
        
\end{itemize}

    \chapter{Pruebas}

\section{Tipos de pruebas en Grails}
Grails divide las pruebas en 2 tipos:
\begin{description}
 \item [Unitarias] La ejecuci\'on de estas pruebas es r\'apida, comparada con las de integraco\'on. La desventaja es que requieren de un uso extensivo de \textit{mocks} y \textit{stubs} para imitar el comportamiento de algunas clases y regresar valores predeterminados para la prueba. 
 \item [Integraci\'on] Estas pruebas necesitan que la aplicaci\'on est\'e cargada por completo para poder realizarse, incluyendo la base de datos, por lo que son m\'as lentas. 
\end{description}

\section{Objetos impl\'icitos en clases de prueba}

Dentro de las clases de prueba existen varios objetos con los que puedes contar para apoyarte durante su implantaci\'on:

\begin{description}
 \item [controller] Es una referencia al controller definido en la anotaci\'on \textit{@TestFor}, lo que permite el acceso a los m\'etodos del controller.
 \item [params] Es una referencia a los par\'ametros enviados al controller para usarlos en su l\'ogica. Al llenar este objeto los \textit{params} ser\'an enviados al objeto \textit{controller} cuando se invoque alguno de sus m\'etodos.
 \item [response] Una vez que alguno de los m\'etodos del \textit{controller} sea invocado, el objeto \textit{response} se llenar\'a con los detalles de la respuesta (estatus, texto de respuesta, headers, etc).
\end{description}


Algunas muestras de c\'odigo para realizar pruebas en Grails:

\begin{itemize}

    \item \textbf{Clase para probar un controller. Autogenerada junto con el controlador \textit{StoreController}}
        \begin{lstlisting}[gobble=11]             
            @TestFor(StoreController)
            class StoreControllerTests {
                void testSomething() {
                    fail "Implement me"
                }
            }
        \end{lstlisting}

    \item \textbf{Clase para probar el texto de respuesta de un controller al ejecutar su m\'etodo \textmd{index()}:}
        \begin{lstlisting}[gobble=11]             
            @TestFor(StoreController)
            class StoreControllerTests {
                void testSomething() {
                    controller.index()
                    assert 'Welcome to the gTunes store!' == response.text
                }
            }        
        \end{lstlisting}

        

\end{itemize}

\section{Comandos \'utiles en el desarrollo de pruebas}
Los resultados de la ejecuci\'on de alg\'un comando de pruebas generan archivos\footnote{Los resultados se generan en el folder test/reports} con los resultados en los formatos \textit{html, txt} y \textit{xml}.
\begin{itemize}

    \item \textbf{Comando que ejecuta todas las pruebas del sistema}
        \begin{lstlisting}[gobble=11]             
            grails test-app
        \end{lstlisting}

    \item \textbf{Clase para probar el texto de respuesta de un controller al ejecutar su m\'etodo \textmd{index()}:}
        \begin{lstlisting}[gobble=11]             
            @TestFor(StoreController)
            class StoreControllerTests {
                void testSomething() {
                    controller.index()
                    assert 'Welcome to the gTunes store!' == response.text
                }
            }        
        \end{lstlisting}

        

\end{itemize}

%     \backmatter
    \appendix    
    % \chapter{Arquitectura y comandos b\'asicos}
\chapter{Comandos m\'as comunes}

\begin{itemize}

    \item \textbf{Consultar versi\'on de Grails}    
        \\Devolver\'a una l\'inea de texto describiendo la versi\'on de grails.
        \begin{lstlisting}[gobble=11] 
            grails -version
        \end{lstlisting}
        

    \item \textbf{Solicitar ayuda de un comando}
        \\Devolver\'a informaci\'on (no tan detallada) sobre el comando y sus argumentos.
        \begin{lstlisting}[gobble=11]
            grails help NOMBRE_DE_COMANDO
        \end{lstlisting}
        

    \item \textbf{Crear aplicaci\'on de Grails}
        \\Genera toda la estructura de archivos de una aplicaci\'on de Grails.
        \begin{lstlisting}[gobble=11]
            grails create-app NOMBRE_DE_APP
        \end{lstlisting}
        

    \item \textbf{Crear un controller + sus pruebas + sus vistas}
        \\Genera un controlador de Grails as\'i como archivos de prueba y \textit{gsp's}. Tanto los archivos de prueba como los \textit{gsp's} se generan vac\'ios.
        \begin{lstlisting}[gobble=11]
            grails create-controller [PAQUETE]NOMBRE
        \end{lstlisting}
        

    \item \textbf{Ejecutar pruebas de la aplicaci\'on}
        \\Ejecuta todas las pruebas de la aplicaci\'on, poniendo los resultados en \textit{test/reports}
        \begin{lstlisting}[gobble=11]
            grails test-app
        \end{lstlisting}
        

    \item \textbf{Ejecutar la aplicaci\'on}
        \\Ejecuta la aplicaci\'on. Si ENVIRONMENT no se especifica, el default a usar ser\'a el de desarrollo (development).
        \begin{lstlisting}[gobble=11]
            grails [ENVIRONMENT] run-app
        \end{lstlisting}
        
        
    \item \textbf{Crear clase de dominio}
        \\Se encarga de crear una nueva clase de dominio y adem\'as genera un archivo donde pueden implantarse sus pruebas.
        \begin{lstlisting}[gobble=11]
            grails create-domain-class [PAQUETE]NOMBRE
        \end{lstlisting}        
        
    \item \textbf{Abrir reporte de pruebas}
        \\Se encarga de abrir el reporte con los resultados de las pruebas con formato HTML en el navegador predeterminado.
        \begin{lstlisting}[gobble=11]
            grails open test-report
        \end{lstlisting}            
\end{itemize}

    % \chapter{Arquitectura y comandos b\'asicos}
\chapter{Inyecci\'on de dependencias en Grails}
\label{_depinject}

    
    \listoffigures
    \listoftables
    %\nocite{*}
    \begin{thebibliography}{99}    
        \bibitem{bib-name1} article-description1
        \bibitem{bib-name2} article-description2
    \end{thebibliography}
\end{document}

%\documentclass[a4paper,10pt]{book}
%\usepackage[utf8]{inputenc}

%\begin{document}

%\end{document}
